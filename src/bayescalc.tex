\documentclass{article}
\title{How likely is the null hypothesis?}
\author{Andy J. Wills}

\begin{document}
\maketitle

In our example, we estimated that the probability the null hypothesis
is true, prior to data collection, was .95. So $P(H_0) = .95$. Our t-test gave
$p = .049$, so $P(data|H_0) = .049$. To work out $P(H_0|data)$, we use Bayes'
theorem. Bayes' is most typically written as:

\begin{equation}
  P(H_0 | data) = \frac{P(data | H_0) P(H_0)} {P(data)}
  \label{bayes}
\end{equation}

but can equivalently be expressed as:

\begin{equation}
 P(H_0 | data) = \frac{P(data | H_0) P(H_0)} {P(data | H_0) P(H_0) +
   P(data | \overline{H_0}) P(\overline{H_0})}
 \label{bayesalt}
\end{equation}

where $P(\overline{H_0})$ means the probability the null hypothesis is
false\footnote{Note $\overline{H_0}$ is a different thing to the probability
  that the experimental hypothesis is true, $P(H_1)$. Null-hypothesis
  significance testing does not give you that number}. Equation~\ref{bayesalt}
is more useful in this case, as $P(\overline{H_0}) = 1 - P(H_0)$ and
$P(data | \overline{H_0}) = 1 - P(data | H_0)$. By substitution, we can see that:

\begin{equation}
P(null | data) =
\frac{0.049 \times .95}
{.049 \times .95 + .951 \times .05} = 0.49
\label{bayeseg}
\end{equation}

which leads to the claim that the probability of the null hypothesis after a
significant t-test is, in this case, close to $50:50$.

\end{document}
%%% Local Variables:
%%% mode: latex
%%% TeX-master: t
%%% End:
